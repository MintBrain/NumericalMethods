\documentclass[10pt]{article}
\usepackage[russian]{babel}
\usepackage[utf8]{inputenc}
\usepackage[T2A]{fontenc}
\usepackage{amsmath}
\usepackage{amsfonts}
\usepackage{amssymb}
\usepackage[version=4]{mhchem}
\usepackage{stmaryrd}

\begin{document}
Сплайн-интерполяция предполагает представление интерполирующей функции в виде комбинации разных функций, соответствующих отрезкам между соседними узлами.
На функции-сплайны накладываются условия непрерывности, т.е. совпадения значений для соседних сплайнов в узле.
Условие непрерывности может касаться как функции, так и ее производных, в зависимости от сложности сплайна.
Из условий непрерывности определяются коэффициенты сплайнов, которые и задают интерполирующую функцию в целом.
Простейший вид сплайн-интерполяции – ступенчатая интерполяция,функции-сплайны постоянны между узлами.
Линейный сплайн непрерывен в узлах интерполяции, первая производная имеет разрывы, вторая и высшие производные не существуют.
Для достижения более высокой точности интерполирования применяют полиномиальные сплайны более высоких степеней.
Наиболее широкое применение получил кубический сплайн.
Кубический сплайн на каждом отрезке между соседними узлами представляет собой полином 3-й степени, удовлетворяет условию непрерывности вместе со своей первой и второй производной.

Сплайн-интерполяция функции $y=y(x)$, заданной таблицей значений в узлах $\left(x_{i} ; y_{i}\right)$, определяет набор фунций-сплайнов $f_{i}(x)$, аппроксимирующих $y(x)$ на интервалах $x_{i-1} \leq x<x_{i}, i=0,1,2, \ldots, n$.

\begin{center}
\begin{tabular}{|c|c|c|}
\hline
$i$ & $x_{i}$ & $y_{i}$ \\
\hline
0 & $x_{0}$ & $y_{0}$ \\
\hline
1 & $x_{1}$ & $y_{1}$ \\
\hline
2 & $x_{2}$ & $y_{2}$ \\
\hline
$\ldots$ & $\ldots$ & $\ldots$ \\
\hline
$n$ & $x_{n}$ & $y_{n}$ \\
\hline
\end{tabular}
\end{center}

$$
y(x)=\left\{\begin{array}{cc}
f_{1}(x) & x_{0} \leq x<x_{1} \\
f_{2}(x) & x_{1} \leq x<x_{2} \\
\ldots & \ldots \\
f_{n}(x) & x_{n-1} \leq x<x_{n}
\end{array}\right.
$$

Если применить кубические сплайны, то


\begin{equation*}
f_{i}(x)=a_{i}+b_{i}\left(x-x_{i}\right)+c_{i}\left(x-x_{i}\right)^{2}+d_{i}\left(x-x_{i}\right)^{3} . \tag{5.4}
\end{equation*}


Введем обозначения $\quad h_{i}=x_{i}-x_{i-1}$, тогда в пределах каждого из сплайнов $0 \leq x-x_{i}<h_{i}$.

Из условия непрерывности функции $f(x)$ следует $2 n$ уравнений:

$$
\begin{aligned}
& f_{i}\left(x_{i-1}\right)=y_{i-1} \\
& f_{i}\left(x_{i}\right)=y_{i} \quad i=1,2, \ldots, n .
\end{aligned}
$$

Или

\[
\begin{array}{rl}
a_{i}=y_{i-1} & \\
a_{i}+b_{i} h_{i}+c_{i} h_{i}^{2}+d_{i} h_{i}^{3}=y_{i} & i=1,2, \ldots, n . \tag{5.6}
\end{array}
\]

Из условия непрерывности 1-й производной функции $f(x)$ следует $n-1$ уравнений:

$$
f_{i}^{\prime}\left(x_{i}\right)=f_{i+1}^{\prime}\left(x_{i}\right) \quad i=1,2, \ldots, n-1
$$

Т.к. $f_{i}^{\prime}=b_{i}+2 c_{i}\left(x-x_{i-1}\right)+3 d_{i}\left(x-x_{i-1}\right)^{2}$, то

\[
\begin{array}{ccc}
b_{i}+2 c_{i}\left(x_{i}-x_{i-1}\right)+3 d_{i}\left(x_{i}-x_{i-1}\right)^{2} & =b_{i+1} & \text { или } \\
b_{i}+2 c_{i} h_{i}+3 d_{i} h_{i}^{2}-b_{i+1}=0 & i=1,2, \ldots, & n-1 . \tag{5.7}
\end{array}
\]

Из условия непрерывности 2-й производной функции $f(x)$ следует $n-1$ уравнений:

$$
f_{i}^{\prime \prime}\left(x_{i}\right)=f_{i+1}^{\prime \prime}\left(x_{i}\right) \quad i=1,2, \ldots, n-1
$$

Т.к. $f_{i}^{\prime \prime}=2 c_{i}+6 d_{i}\left(x-x_{i-1}\right)$, то

\[
\begin{array}{rl}
2 c_{i}+6 d_{i}\left(x_{i}-x_{i-1}\right)=2 c_{i+1} & \text { или } \\
c_{i}+3 d_{i} h_{i}-c_{i+1}=0 & i=1,2, \ldots, n-1 . \tag{5.8}
\end{array}
\]

Получаем $2 n+(n-1)+(n-1)=4 n-2$ уравнения относительно $4 n$ неизвестных. Оставшиеся два уравнения задают, фиксируя значения производных на концах кривой, например так:

$$
f^{\prime \prime}\left(x_{0}\right)=0 \quad f^{\prime \prime}\left(x_{n}\right)=0,
$$

или


\begin{gather*}
c_{1}=0  \tag{5.9}\\
c_{n}+3 d_{n} h_{n}=0 \tag{5.10}
\end{gather*}


Полученные уравнения представляют собой систему линейных алгебраических уравнений относительно $4 n$ неизвестных $a_{i}, b_{i}, c_{i}, d_{i}$, $(i=0,1, \ldots, n)$.

Эту систему можно привести к более удобному виду. Из условия (5.5) сразу можно найти все коэффициенты $a_{i}$. Далее из (5.8)-(5.10) получим


\begin{equation*}
d_{i}=\frac{c_{i+1}-c_{i}}{3 h_{i}}, \quad i=1,2, \ldots, n-1, \quad d_{n}=-\frac{c_{n}}{3 h_{n}} \tag{5.11}
\end{equation*}


Подставим эти соотношения, а также значения $a_{i}=y_{i-1}$ в (5.6) и найдем отсюда коэффициенты


\begin{align*}
& b_{i}=\frac{y_{i}-y_{i-1}}{h_{i}}-\frac{h_{i}}{3}\left(c_{i+1}+2 c_{i}\right), \quad i=1,2, \ldots, n-1, \\
& b_{n}=\frac{y_{n}-y_{n-1}}{h_{n}}-\frac{2}{3} h_{n} c_{n} . \tag{5.12}
\end{align*}


Учитывая выражения (5.11) и (5.12), исключаем из уравнения (5.7) коэффициенты $d_{i}$ и $b_{i}$. Окончательно получим следующую систему уравнений только для коэффициентов $c_{i}$ :\\
$c_{1}=0, \quad c_{n+1}=0$,\\
$h_{i-1} c_{i-1}+2\left(h_{i-1}+h_{i}\right) c_{i}+h_{i} c_{i+1}=3\left(\frac{y_{i}-y_{i-1}}{h_{i}}-\frac{y_{i-1}-y_{i-2}}{h_{i-1}}\right)$,\\
$i=2,3, \ldots, n$.\\
Матрица этой системы трехдиагональная, т.е. ненулевые элементы находятся лишь на главной и двух соседних с ней диагоналях, расположенных сверху и снизу.
Для ее решения целесообразно использовать метод прогонки.
По найденным из системы (5.13) коэффициентам $c_{i}$ легко вычислить коэффициенты $d_{i,} b_{i}$.

\end{document}